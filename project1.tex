\documentclass[12pt,a4paper]{article}
\usepackage{color}
\usepackage{float}
\usepackage{graphicx}
\usepackage{indentfirst}
\usepackage{amsmath}
\usepackage{multirow}
\def\degree{${}^{\circ}$}
\begin{document}

\vspace*{0.25cm}

\hrulefill

\thispagestyle{empty}

\begin{center}
\begin{large}
\sc{UM--SJTU Joint Institute \vspace{0.3em} \\ Probabilistic Methods in Engineering \\(VE401)}
\end{large}

\hrulefill

\vspace*{5cm}
\begin{Large}
\textbf{{Term Project \uppercase\expandafter{\romannumeral1}}}
\end{Large}

\vspace{2em}

\begin{large}
\textbf{A Quest for Randomness\\
\vspace{0.8em}
 }
\end{large}
\end{center}


\vfill

\begin{table}[h!]
\flushleft
\begin{tabular}{lll}
Project Group 27: & \\
&\\
Name: Liu Niyiqiu \hspace*{2em}&
ID: 516370910118\hspace*{2em}
\\
Name: Liu Niyiqiu \hspace*{2em}&
ID: 516370910118\hspace*{2em}
\\
Name: Liu Niyiqiu \hspace*{2em}&
ID: 516370910118\hspace*{2em}
\\
Name: Liu Niyiqiu \hspace*{2em}&
ID: 516370910118\hspace*{2em}
\\
Name: Liu Niyiqiu \hspace*{2em}&
ID: 516370910118\hspace*{2em}
\\


\\

Date: 7 March 2018
\end{tabular}
\end{table}

\hfill

\newpage

\begin{large}
	\begin{center}
		\textbf{Abstract}
	\end{center}
\end{large}
Abstract here.
\\\\
\textbf{Keyword:} 
\\\\
\tableofcontents

\newpage
\section{Introduction}
\section{Problems with Linear Congruential Generator}
\subsection{Linear Congruential Generator}
A linear congruential generator (LCG) is an algorithm that yields a sequence of pseudo-randomized numbers calculated with a discontinuous piecewise linear equation. It is an old and widely used algorithm to study Monte-Carlo problems. 


The principle behind this algorithm is intuitive. The sequence of random numbers $r_i$ is determined by the recurrence relation listed in Equation \ref{r}.  
\begin{equation}
r_{n+1} = (a\cdot r_n + c) \quad \text{mod} \quad m
\label{r}
\end{equation}

The multiplier $a$, increment $c$, modulus $m$ and start value $r_0$ are chosen appropriately so that a long and known period can be achieved. The common values for these constants are listed in Table \ref{tab:table1}.

\begin{table}[http]
	\begin{tabular}{|c|c|c|c|}
		\hline 
		Source &Modulus $m$ & Multiplier $a$ & Increment $c$  \\ 
		\hline 
		Numerical Recipes& $2^{32}$ &1664525  &1013904223  \\ 
		\hline 
		Borland C/C++&$2^{32}$  &22695477  &1  \\ 
		\hline 
		glibc(used by GCC)&$2^{31}$ &1103515245  &12345  \\ 
		\hline 
		Microsoft Visual/Quick C&$2^{32}$  &  214013($343FD_{16}$)& 2531011($269EC3_{16}$)  \\ 
		\hline 
		C++11's minstd\_rand& $2^{31} - 1$ &48271  &0  \\ 
		\hline 
		JAVA's java.util.Random& $2^{48}$ &25214903917  &11  \\ 
		\hline 
		Newlib, Musl&$2^{64}$  &6364136223846793005  &1  \\ 
		\hline 
	\end{tabular}
	\caption{Common values used in a linear congruential generator.[2]}
	\label{tab:table1} 
\end{table}


\subsection{Example of a Linaer Congruential Generator}

\subsection{Marsaglia's Theorem}
One problem with the linear congruential generator is that the numbers generated by it exhibit a particular structure. This result was discovered by statistician George Marsaglia in 1968 and is known as Marsaglia's Theorem.

Marsaglia's Theorem states that if $n$-tuples ($u_1$, $u_2$, $\cdots$, $u_n$), ($u_2$, $u_3$, $\cdots$, $u_{n+1}$), $\cdots$ of uniform variates produced by the generator are viewed as points in the unit cube of $n$ dimensions, then all the points will be found to lie in a relatively small number of parallel hyperplanes. Furthermore, there are many systems of parallel hyperplanes which contain all of the points and the number of hyperplanes are bounded by a known value.



\subsection{Proof for Marsaglia's Theorem}
\subsubsection{Nomenclature}
For any modulus $m$ and multiplier $k$, let 
\begin{equation*}
r1, r2, r3 \cdots \quad 0 < r_i < m
\end{equation*}
be a sequence of residues of $M$ generated by the recurrence relation
\begin{equation*}
r_{i+1} \equiv k r_i \text\quad {modulo}\quad m,
\end{equation*}
and let $u_1$, $u_2$, $u_3$, $\cdots$ be that sequence viewed as fractions of $m$:
\begin{equation*}
u_1 = r_1/m, \quad u_2 = r_2/m, \quad u_3 = r_3/m, \cdots
\end{equation*}
Let $\pi_1$ = ($u_1$, $\cdots$, $u_n$), $\pi_2$ = ($u_2$, $\cdots$, $u_{n+1}$), $\pi_3$ = ($u_3$, $\cdots$, $u_{n+2}$), $\cdots$ be points of the unit $n$-cube formed from $n$ successive $u$'s. 


\subsubsection{Theorem}
If $c_1$, $c_2$, $\cdots$, $c_n$ is any choice of integers such that 
\begin{equation*}
c_1 + c_2k + c_3 k^2 + \cdots + c_n k^{n-1}\equiv 0 \quad \text{modulo} \quad m
\end{equation*}
then all of the points $\pi_1$, $\pi_2$, $\cdots$ will lie in the set of parallel hyperplanes defined by the equations 
\begin{equation*}
c_1x_1 + c_2x_2 + \cdots + c_n x_n = 0, \pm 1, \pm 2, \cdots
\end{equation*}
There are at most 
\begin{equation*}
|c_1| + |c_2| + \cdots +|c_n| 
\end{equation*}
of these hyperplanes which intersect the unit $n$-cube, and there is always a choice of $c_1$, $c_2$, $\cdots$, $c_n$ such that all of the points fall in fewer than $(n!m)^{1/n}$ hyperplanes.



\subsubsection{Proof}
We prove Marsaglia's Theorem through four steps.


\noindent{\textbf{Step 1:} If}
\begin{equation}
c_1 + c_2k + c_3 k^2 + \cdots + c_n k^{n-1}\equiv 0 \quad \text{modulo} \quad m
\label{eq:assumption}
\end{equation}
then 
\begin{equation*}
c_1u_i + c_2u_{i+1} + \cdots + c_nu_{i+n-1}
\end{equation*}
is an integer for every $i$.


\noindent{\textbf{Proof: }} 
\\
By the definition of the modulus operation, the sequence $r_1$, $r_2$, $\cdots$ can be put in the form
\begin{equation*}
\frac{r_1}{m} - m\Big[\frac{r_1}{m}\Big], \frac{kr_1}{m} - m\Big[\frac{kr_1}{m}\Big], \frac{k^2r_1}{m} - m\Big[\frac{k^2r_1}{m}\Big], \frac{k^3r_1}{m} - m\Big[\frac{k^3r_1}{m}\Big], \cdots 
\end{equation*}
where $[\cdot]$ is the greatest integer notation.
\\
Then, we can write $u_1$, $u_2$, $\cdots$ as 
$$
\frac{r_1}{m} - \Big[\frac{r_1}{m}\Big], \frac{kr_1}{m} - \Big[\frac{kr_1}{m}\Big], \frac{k^2r_1}{m} - \Big[\frac{k^2r_1}{m}\Big], \frac{k^3r_1}{m} - \Big[\frac{k^3r_1}{m}\Big], \cdots 
$$
\\
To consider whether $c_1u_i + \cdots + c_nu_{i+n-1}$ is an integer, we only need to consider the first term in $u_i$, $k^ir_1/m$, since the second term is already an integer. Then
\begin{align*}
&\qquad c_1\frac{k^ir_1}{m} + c_{2}\frac{k^{i+1}r_1}{m} + \cdots + c_{n}\frac{k^{i+n}r_1}{m}\\
&=\frac{k^ir_1}{m}(c_1 + c_2k + \cdots + c_nk^{n-1})
\end{align*}
would be an integer by Equation \ref{eq:assumption}. Hence step 1 is verified.
\\\\

\noindent{\textbf{Step 2:} }
The points $\pi_i$ $=$ ($u_i$, $u_{i+1}$, $\cdots$, $u_{i+n-1}$) must line in one of the hyperplanes
$$
c_1x_1 + c_2x_2 + \cdots + c_nx_n = 0, \pm1, \pm2, \pm3, \cdots
$$
\noindent{\textbf{Proof: }} 
This result is obtained from the fact that $c_1u_i + c_2u_{i+1} + \cdots + c_nu_{i+n-1}$ is an integer for every $i$ provided that Equation \ref{eq:assumption} holds.
\\\\
\noindent{\textbf{Step 3:} }
The number of hyperplanes of the above type which intersect the unit $n$-cube, $0<x_1<1$, $\cdots$, $0 < x_n < 1$, is at most
$$
|c_1| + |c_2| + \cdots + |c_n|,
$$
\noindent{\textbf{Proof: }} 
\begin{align*}
c_1x_1 + c_2x_2 + \cdots + c_nx_n &\leq |c_1x_1| + |c_2x_2| + \cdots + |c_nx_n|\\
&\leq  |c_1| + |c_2| + \cdots + |c_n|
\end{align*}
\\\\
\noindent{\textbf{Step 4:} }
For every multiplier $k$ and modulus $m$ there 
\section{Conclusions and Discussion}
\section{References}
\begin{enumerate}
	\item George Marsaglia. Random numbers fall mainly in the planes. \textit{Proceedings of the National Academy of
		Sciences of the United States of America}, 61(1): 25–28, 1968.
	\item Wikipedia contributors. "Linear congruential generator." \textit{Wikipedia, The Free Encyclopedia}. Wikipedia, The Free Encyclopedia, 24 Feb. 2018. Web. 11 Mar. 2018.
	
	
\end{enumerate}
\end{document}
